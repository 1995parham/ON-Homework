\documentclass[paper=a4, fontsize=11pt]{article}

%----------------------------------------------------------------------------------------
%	PACKAGES AND OTHER DOCUMENT CONFIGURATIONS
%----------------------------------------------------------------------------------------

\usepackage{amsmath,amsfonts,amsthm} % Math packages
\usepackage{sectsty} % Allows customizing section commands
\allsectionsfont{\centering \normalfont\scshape} % Make all sections centered, the default font and small caps

\usepackage{fancyhdr} % Custom headers and footers
\pagestyle{fancyplain} % Makes all pages in the document conform to the custom headers and footers
\fancyhead{} % No page header - if you want one, create it in the same way as the footers below
\fancyfoot[L]{} % Empty left footer
\fancyfoot[C]{} % Empty center footer
\fancyfoot[R]{\thepage} % Page numbering for right footer
\renewcommand{\headrulewidth}{0pt} % Remove header underlines
\renewcommand{\footrulewidth}{0pt} % Remove footer underlines
\setlength{\headheight}{13.6pt} % Customize the height of the header

\numberwithin{equation}{section} % Number equations within sections (i.e. 1.1, 1.2, 2.1, 2.2 instead of 1, 2, 3, 4)
\numberwithin{figure}{section} % Number figures within sections (i.e. 1.1, 1.2, 2.1, 2.2 instead of 1, 2, 3, 4)
\numberwithin{table}{section} % Number tables within sections (i.e. 1.1, 1.2, 2.1, 2.2 instead of 1, 2, 3, 4)

\setlength\parindent{0pt} % Removes all indentation from paragraphs - comment this line for an assignment with lots of text

\usepackage{graphicx}
\graphicspath{ {images/} }

\usepackage{xepersian}
\settextfont[Path=fonts/]{Vazir.ttf}
\setlatintextfont{Times New Roman}

%----------------------------------------------------------------------------------------
%	TITLE SECTION
%----------------------------------------------------------------------------------------

\newcommand{\horrule}[1]{\rule{\linewidth}{#1}} % Create horizontal rule command with 1 argument of height

\title{
\normalfont\normalsize
\includegraphics[scale=0.1]{aut}
\hspace{5cm}
\includegraphics[scale=0.1]{ceit} \\
\textsc دانشگاه صنعتی امیرکبیر \\
\textsc دانشکده مهندسی کامپیوتر و فناوری اطلاعات
\horrule{0.5pt} \\ [0.4cm] % Thin top horizontal rule
\huge بهینه‌سازی و کاربرد آن در شبکه‌های کامپیوتری \\
\huge تمرین اول
\horrule{2pt} \\ [0.5cm] % Thick bottom horizontal rule
}

\author{پرهام الوانی}

\date{\normalsize\today} % Today's date or a custom date

\begin{document}

\maketitle % Print the title

\section{مدل‌سازی}
\paragraph{
    در ابتدا مساله را مدل‌سازی می‌کنیم، در این مدل‌سازی محدودیت‌های لینک‌ها و اولویت‌های کاربران را نیز مدنظر قرار می‌دهیم.
}

\begin{equation}
	\begin{aligned}
		& \underset{x}{\text{max}}
		& & \sum_{i=1}^{3} x_i \\
		& \text{s.t.} \\
		& & & x_1 \le 20 \\
		& & & x_1 + x_2 \le 30 \\
		& & & x_2 \le 20 \\
		& & & x_2 + x_3 \le 30 \\
		& & & x_3 \le 25 \\
		& & & x_2 \le \log(x_1) \\
	\end{aligned}					
\end{equation}

\paragraph{
    مدل حاصل را به فرم استاندارد بازنویسی می‌کنیم.
}

\begin{equation}
	\begin{aligned}
		& \underset{x}{\text{min}}
		& & -\sum_{i=1}^{3} x_i \\
		& \text{s.t.} \\
		& & & x_1 - 20 \le 0 \\
		& & & x_1 + x_2 - 30 \le 0 \\
		& & & x_2 - 20 \le 0 \\
		& & & x_2 + x_3 - 30 \le 0 \\
		& & & x_3 - 25 \le 0 \\
		& & & x_2 - \log(x_1) \le 0 \\
	\end{aligned}					
\end{equation}

\section{حذف محدودیت‌ها}

\paragraph{
    از آنجایی که مدل حاصل تنها محدودیت‌های نامساوی دارد از \lr{barrier} استفاده می‌کنیم و مدل را بازنویسی می‌کنیم.
}

\begin{equation}
	\begin{aligned}
		& \underset{x}{\text{min}}
		& & -\sum_{i=1}^{3} x_i \\
		& & & - \mu\frac{1}{x_1 - 20} \\
		& & & - \mu\frac{1}{x_1 + x_2 - 30} \\
		& & & - \mu\frac{1}{x_2 - 20} \\
		& & & - \mu\frac{1}{x_2 + x_3 - 30} \\
		& & & - \mu\frac{1}{x_3 - 25} \\
		& & & - \mu\frac{1}{x_2 - \log(x_1)} \\
	\end{aligned}					
\end{equation}

\section{جستجوی خطی}
\paragraph{
    الگوریتم جستجوی خطی مبتنی بر \lr{backtracking} و \lr{steepest descent} با زبان \lr{go}
    پیاده‌سازی شد.
}

\end{document}
