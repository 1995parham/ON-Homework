\documentclass[paper=a4, fontsize=11pt]{article}

%----------------------------------------------------------------------------------------
%	PACKAGES AND OTHER DOCUMENT CONFIGURATIONS
%----------------------------------------------------------------------------------------

\usepackage{amsmath,amsfonts,amsthm} % Math packages
\usepackage{sectsty} % Allows customizing section commands
\allsectionsfont{\centering \normalfont\scshape} % Make all sections centered, the default font and small caps

\usepackage{fancyhdr} % Custom headers and footers
\pagestyle{fancyplain} % Makes all pages in the document conform to the custom headers and footers
\fancyhead{} % No page header - if you want one, create it in the same way as the footers below
\fancyfoot[L]{} % Empty left footer
\fancyfoot[C]{} % Empty center footer
\fancyfoot[R]{\thepage} % Page numbering for right footer
\renewcommand{\headrulewidth}{0pt} % Remove header underlines
\renewcommand{\footrulewidth}{0pt} % Remove footer underlines
\setlength{\headheight}{13.6pt} % Customize the height of the header

\numberwithin{equation}{section} % Number equations within sections (i.e. 1.1, 1.2, 2.1, 2.2 instead of 1, 2, 3, 4)
\numberwithin{figure}{section} % Number figures within sections (i.e. 1.1, 1.2, 2.1, 2.2 instead of 1, 2, 3, 4)
\numberwithin{table}{section} % Number tables within sections (i.e. 1.1, 1.2, 2.1, 2.2 instead of 1, 2, 3, 4)

\setlength\parindent{0pt} % Removes all indentation from paragraphs - comment this line for an assignment with lots of text

\usepackage{graphicx}
\graphicspath{ {images/} }

\usepackage{xepersian}
\settextfont[Path=fonts/]{Vazir.ttf}
\setlatintextfont{Times New Roman}

%----------------------------------------------------------------------------------------
%	TITLE SECTION
%----------------------------------------------------------------------------------------

\newcommand{\horrule}[1]{\rule{\linewidth}{#1}} % Create horizontal rule command with 1 argument of height

\title{
\normalfont\normalsize
\includegraphics[scale=0.1]{aut}
\hspace{5cm}
\includegraphics[scale=0.1]{ceit} \\
\textsc دانشگاه صنعتی امیرکبیر \\
\textsc دانشکده مهندسی کامپیوتر و فناوری اطلاعات
\horrule{0.5pt} \\ [0.4cm] % Thin top horizontal rule
\huge بهینه‌سازی و کاربرد آن در شبکه‌های کامپیوتری \\
\huge تمرین اول
\horrule{2pt} \\ [0.5cm] % Thick bottom horizontal rule
}

\author{پرهام الوانی}

\date{\normalsize\today} % Today's date or a custom date

\begin{document}

\maketitle % Print the title

\section{مدل‌سازی}
\paragraph{
    در ابتدا مساله را مدل‌سازی می‌کنیم، در این مدل‌سازی محدودیت‌های لینک‌ها و اولویت‌های کاربران را نیز مدنظر قرار می‌دهیم.
}

\begin{equation}
	\begin{aligned}
		& \underset{x}{\text{max}}
		& & \sum_{i=1}^{3} x_i \\
		& \text{s.t.} \\
		& & & x_1 \le 20 \\
		& & & x_1 + x_2 \le 30 \\
		& & & x_2 \le 20 \\
		& & & x_2 + x_3 \le 30 \\
		& & & x_3 \le 25 \\
		& & & x_2 \le \log(x_1) \\
	\end{aligned}					
\end{equation}

\paragraph{
    مدل حاصل را به فرم استاندارد بازنویسی می‌کنیم.
}

\begin{equation}
	\begin{aligned}
		& \underset{x}{\text{min}}
		& & -\sum_{i=1}^{3} x_i \\
		& \text{s.t.} \\
		& & & x_1 - 20 \le 0 \\
		& & & x_1 + x_2 - 30 \le 0 \\
		& & & x_2 - 20 \le 0 \\
		& & & x_2 + x_3 - 30 \le 0 \\
		& & & x_3 - 25 \le 0 \\
		& & & x_2 - \log(x_1) \le 0 \\
	\end{aligned}					
\end{equation}

\section{حذف محدودیت‌ها}

\paragraph{
    از آنجایی که مدل حاصل تنها محدودیت‌های نامساوی دارد از \lr{barrier} استفاده می‌کنیم و مدل را بازنویسی می‌کنیم.
}

\begin{equation}
	\begin{aligned}
		& \underset{x}{\text{min}}
		& & -\sum_{i=1}^{3} x_i \\
		& & & - \mu\frac{1}{x_1 - 20} \\
		& & & - \mu\frac{1}{x_1 + x_2 - 30} \\
		& & & - \mu\frac{1}{x_2 - 20} \\
		& & & - \mu\frac{1}{x_2 + x_3 - 30} \\
		& & & - \mu\frac{1}{x_3 - 25} \\
		& & & - \mu\frac{1}{x_2 - \log(x_1)} \\
	\end{aligned}					
\end{equation}

\section{جستجوی خطی}
\paragraph{
    الگوریتم جستجوی خطی مبتنی بر \lr{backtracking} و \lr{steepest descent} با زبان \lr{go}
	پیاده‌سازی شد.
	در ادامه ورودی‌ها و خروجی برنامه را مرور می‌کنیم.
}

\begin{center}
	ورودی‌ها\\
	\begin{tabular}{| l | c |}
		\hline
		$(11, 1, 1)$ & $x_0$ \\
		$1$ & $\alpha$ \\
		$0.5$ & $\beta$ \\
		$0.5$ & $c$ \\
		$0.001$ & $\epsilon$ \\
		$1$ & $\mu$ \\
		\hline
	\end{tabular}
\end{center}

\begin{center}
	نتایج اولین اجرا\\
	\begin{tabular}{| l | c |}
		\hline
		تعداد مراحل اجرا & 28 \\
		جواب بهینه & 53.41- \\
		\hline
	\end{tabular}
\end{center}

\paragraph{
	در صورتی که مقدار $\mu$ را کاهش دهیم جواب‌های بهتری بدست خواهد آمد که در ادامه تعدادی از آن‌ها را می‌بینیم.
}

\begin{center}
	ورودی‌ها\\
	\begin{tabular}{| l | c |}
		\hline
		$(11, 1, 1)$ & $x_0$ \\
		$1$ & $\alpha$ \\
		$0.5$ & $\beta$ \\
		$0.5$ & $c$ \\
		$0.001$ & $\epsilon$ \\
		$0.5$ & $\mu$ \\
		\hline
	\end{tabular}
\end{center}

\begin{center}
	نتایج دومین اجرا\\
	\begin{tabular}{| l | c |}
		\hline
		تعداد مراحل اجرا & 75 \\
		جواب بهینه & 49.43- \\
		\hline
	\end{tabular}
\end{center}

\begin{center}
	ورودی‌ها\\
	\begin{tabular}{| l | c |}
		\hline
		$(11, 1, 1)$ & $x_0$ \\
		$1$ & $\alpha$ \\
		$0.5$ & $\beta$ \\
		$0.5$ & $c$ \\
		$0.001$ & $\epsilon$ \\
		$0.3$ & $\mu$ \\
		\hline
	\end{tabular}
\end{center}

\begin{center}
	نتایج سومین اجرا\\
	\begin{tabular}{| l | c |}
		\hline
		تعداد مراحل اجرا & 88 \\
		جواب بهینه & 53.44- \\
		\hline
	\end{tabular}
\end{center}

\section{مشکلات}
\paragraph{
	در ابتدا به دنبال تابعی بودم که بتوان به وسیله‌ی آن گرادیان را محاسبه کرد ولی نتوانستم آن را برای زبانی
	که می‌خواستم پیاده‌سازی را برای آن انجام دهم، پیدا کنم.بنابراین گرادیان را به صورت دستی محاسبه کردم و آن را در کد قرار دادم.
}

\section{حل مساله با ktt}
\paragraph{
	ابتدا شرایط kkt را برای مساله می‌نویسیم، از آنجایی که مساله بهینه‌سازی محدب است پس شرایط ktt برای آن لازم و کافی است.
	بنابراین با حل ktt نقطه‌ی بهینه بدست خواهد آمد.
}

\begin{equation}
	\begin{aligned}
		& \exists \lambda_i \\
		& \text{s.t.} \\
		& \left( \begin{array}{c} -1 \\ -1 \\ -1 \end{array} \right)
		+ \lambda_1 \left( \begin{array}{c} 1 \\ 0 \\ 0 \end{array} \right)
		+ \lambda_2 \left( \begin{array}{c} 1 \\ 1 \\ 0 \end{array} \right)\\
		& + \lambda_3 \left( \begin{array}{c} 0 \\ 1 \\ 0 \end{array} \right)
		+ \lambda_4 \left( \begin{array}{c} 0 \\ 1 \\ 1 \end{array} \right)
		+ \lambda_5 \left( \begin{array}{c} 0 \\ 0 \\ 1 \end{array} \right)
		+ \lambda_6 \left( \begin{array}{c} -\frac{1}{x^*_1} \\ 1 \\ 0 \end{array} \right)
		= 0 \\
		& \text{\lr{feasibility conditions}} \\
		& x^*1 - 20 \le 0\\
		& x^*_1 + x^*_2 - 30 \le 0 \\
		& x^*_2 - 20 \le 0 \\
		& x^*_2 + x^*_3 - 30 \le 0 \\
		& x^*_3 - 25 \le 0 \\
		& x^*_2 - \log(x^*_1) \le 0 \\ 
		& \text{\lr{dual feasibility}} \\
		& \lambda_1 \ge 0,
		\lambda_2 \ge 0,
		\lambda_3 \ge 0 \\
		& \lambda_4 \ge 0,
		\lambda_5 \ge 0,
		\lambda_6 \ge 0 \\
		& \text{\lr{complementary slackness}} \\
		& \lambda_1(x^*_1 - 20) = 0\\
		& \lambda_2(x^*_1 + x^*_2 - 30) = 0 \\
		& \lambda_3(x^*_2 - 20) = 0 \\
		& \lambda_4(x^*_2 + x^*_3 - 30) = 0 \\
		& \lambda_5(x^*_3 - 25) = 0 \\
		& \lambda_6(x^*_2 - \log(x^*_1)) = 0 \\
	\end{aligned}					
\end{equation}

\paragraph{
	برای حل فرض می‌کنیم که مقدار $x^*_1$ و $x^*_3$ به ترتیب برابر با $20$ و $25$ می‌باشند. به این ترتیب خواهیم داشت:
}

\begin{equation}
	\begin{aligned}
		& x^*_1 = 20, x_2 = \log(x^*_1) = 2.99, x^*_3 = 25 \\
		& \lambda_1 = 1, \lambda_2 = 0 \\
		& \lambda_3 = 1, \lambda_4 = 0 \\
		& \lambda_5 = 1, \lambda_6 = 0 \\
		& f(x^*) = -20 + -25 + -2.99 = -47.99
	\end{aligned}					
\end{equation}



\end{document}
