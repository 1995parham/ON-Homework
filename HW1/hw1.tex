%----------------------------------------------------------------------------------------
%	PACKAGES AND OTHER DOCUMENT CONFIGURATIONS
%----------------------------------------------------------------------------------------

\documentclass[paper=a4, fontsize=11pt]{scrartcl} % A4 paper and 11pt font size

\usepackage[T1]{fontenc} % Use 8-bit encoding that has 256 glyphs
\usepackage[english]{babel} % English language/hyphenation
\usepackage{amsmath,amsfonts,amsthm} % Math packages

\usepackage{sectsty} % Allows customizing section commands
\allsectionsfont{\centering \normalfont\scshape} % Make all sections centered, the default font and small caps

\usepackage{fancyhdr} % Custom headers and footers
\pagestyle{fancyplain} % Makes all pages in the document conform to the custom headers and footers
\fancyhead{} % No page header - if you want one, create it in the same way as the footers below
\fancyfoot[L]{} % Empty left footer
\fancyfoot[C]{} % Empty center footer
\fancyfoot[R]{\thepage} % Page numbering for right footer
\renewcommand{\headrulewidth}{0pt} % Remove header underlines
\renewcommand{\footrulewidth}{0pt} % Remove footer underlines
\setlength{\headheight}{13.6pt} % Customize the height of the header

\numberwithin{equation}{section} % Number equations within sections (i.e. 1.1, 1.2, 2.1, 2.2 instead of 1, 2, 3, 4)
\numberwithin{figure}{section} % Number figures within sections (i.e. 1.1, 1.2, 2.1, 2.2 instead of 1, 2, 3, 4)
\numberwithin{table}{section} % Number tables within sections (i.e. 1.1, 1.2, 2.1, 2.2 instead of 1, 2, 3, 4)

\setlength\parindent{0pt} % Removes all indentation from paragraphs - comment this line for an assignment with lots of text

%----------------------------------------------------------------------------------------
%	TITLE SECTION
%----------------------------------------------------------------------------------------

\newcommand{\horrule}[1]{\rule{\linewidth}{#1}} % Create horizontal rule command with 1 argument of height

\title{	
	\normalfont\normalsize
	\textsc{Computer Engineering and Information Technology Department of Amirkabir University of Technology} \\ [25pt] % Your university, school and/or department name(s)
	\horrule{0.5pt} \\[0.4cm] % Thin top horizontal rule
	\huge Optimization in Computer Networks \\ % The assignment title
	\horrule{2pt} \\[0.5cm] % Thick bottom horizontal rule
}

\author{Parham Alvani} % Your name

\date{\normalsize\today} % Today's date or a custom date

\begin{document}

\maketitle % Print the title

%----------------------------------------------------------------------------------------
%	PROBLEM 1
%----------------------------------------------------------------------------------------

\section{NUM Problem}

\begin{equation}
	\begin{aligned}
		& \underset{x}{\text{max}}
		& & \sum_{i=1}^{3} x_i \\
		& \text{s.t.} \\
		& & & x_1 \le 20 \\
		& & & x_1 + x_2 \le 30 \\
		& & & x_2 \le 20 \\
		& & & x_2 + x_3 \le 30 \\
		& & & x_3 \le 25 \\
		& & & x_2 \le \log(x_1) \\
	\end{aligned}					
\end{equation}

%------------------------------------------------

\subsection{Standard Form}

\begin{equation}
	\begin{aligned}
		& \underset{x}{\text{max}}
		& & \sum_{i=1}^{3} x_i \\
		& \text{s.t.} \\
		& & & x_1 - 20 \le 0 \\
		& & & x_1 + x_2 - 30 \le 0 \\
		& & & x_2 - 20 \le 0 \\
		& & & x_2 + x_3 - 30 \le 0 \\
		& & & x_3 - 25 \le 0 \\
		& & & x_2 - \log(x_1) \le 0 \\
	\end{aligned}					
\end{equation}

%------------------------------------------------

\subsubsection{Barriers}

\begin{equation}
	\begin{aligned}
		& \underset{x}{\text{max}}
		& & \sum_{i=1}^{3} x_i \\
		& & & + \mu\frac{1}{x_1 - 20} \\
		& & & + \mu\frac{1}{x_1 + x_2 - 30} \\
		& & & + \mu\frac{1}{x_2 - 20} \\
		& & & + \mu\frac{1}{x_2 + x_3 - 30} \\
		& & & + \mu\frac{1}{x_3 - 25} \\
		& & & + \mu\frac{1}{x_2 - \log(x_1)} \\
	\end{aligned}					
\end{equation}

%------------------------------------------------

\end{document}
