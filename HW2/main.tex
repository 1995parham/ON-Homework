\documentclass[paper=a4, fontsize=11pt]{article}

%----------------------------------------------------------------------------------------
%	PACKAGES AND OTHER DOCUMENT CONFIGURATIONS
%----------------------------------------------------------------------------------------

\usepackage{amsmath,amsfonts,amsthm} % Math packages
\usepackage{sectsty} % Allows customizing section commands
\allsectionsfont{\centering \normalfont\scshape} % Make all sections centered, the default font and small caps

\usepackage{fancyhdr} % Custom headers and footers
\pagestyle{fancyplain} % Makes all pages in the document conform to the custom headers and footers
\fancyhead{} % No page header - if you want one, create it in the same way as the footers below
\fancyfoot[L]{} % Empty left footer
\fancyfoot[C]{} % Empty center footer
\fancyfoot[R]{\thepage} % Page numbering for right footer
\renewcommand{\headrulewidth}{0pt} % Remove header underlines
\renewcommand{\footrulewidth}{0pt} % Remove footer underlines
\setlength{\headheight}{13.6pt} % Customize the height of the header

\numberwithin{equation}{section} % Number equations within sections (i.e. 1.1, 1.2, 2.1, 2.2 instead of 1, 2, 3, 4)
\numberwithin{figure}{section} % Number figures within sections (i.e. 1.1, 1.2, 2.1, 2.2 instead of 1, 2, 3, 4)
\numberwithin{table}{section} % Number tables within sections (i.e. 1.1, 1.2, 2.1, 2.2 instead of 1, 2, 3, 4)

\setlength{\parindent}{4em}
\setlength{\parskip}{1em}
\renewcommand{\baselinestretch}{1.5}

\usepackage{graphicx}
\graphicspath{ {images/} }

\usepackage{xepersian}
\settextfont[Path=fonts/]{Vazir.ttf}
\setlatintextfont{Times New Roman}

%----------------------------------------------------------------------------------------
%	TITLE SECTION
%----------------------------------------------------------------------------------------

\newcommand{\horrule}[1]{\rule{\linewidth}{#1}} % Create horizontal rule command with 1 argument of height

\title{
\normalfont\normalsize
\includegraphics[scale=0.1]{aut}
\hspace{5cm}
\includegraphics[scale=0.1]{ceit} \\
\textsc دانشگاه صنعتی امیرکبیر \\
\textsc دانشکده مهندسی کامپیوتر و فناوری اطلاعات
\horrule{0.5pt} \\ [0.4cm] % Thin top horizontal rule
\huge بهینه‌سازی و کاربرد آن در شبکه‌های کامپیوتری \\
\huge تمرین دوم 
\horrule{2pt} \\ [0.5cm] % Thick bottom horizontal rule
}

\author{پرهام الوانی}

\date{\normalsize\today} % Today's date or a custom date

\begin{document}

\maketitle % Print the title

\section{سوال اول}
\subsection{الف}
\paragraph{}
این مجموعه یک مجموعه محدب نیست و برای نشان دادن این موضوع از مثال نقض استفاده می‌کنیم.

\begin{align}
\begin{split}
x_1 = \left[\begin{array}{c} 0 \\ 0 \end{array}\right] \\
x_2 = \left[\begin{array}{c} -8 \\ -2 \end{array}\right] \\
\frac{1}{2} * x_1 + \frac{1}{2} * x_2 \\
&= \left[\begin{array}{c} -4 \\ -1 \end{array}\right] \notin A\\
\end{split}
\end{align}

\subsection{ب}
\paragraph{}
این مجموعه محدب است زیرا تابع:

\begin{align}
\begin{split}
	\left\{
	\begin{array}{lr}
		\lambda_1^3 & \lambda_1 \geq 2\\
		-\lambda_1 + 10 & \lambda_1 < 2\\
	\end{array}
	\right.
\end{split}
\end{align}

\paragraph{}
یک تابع محدب بوده بنابراین مجموعه \(B\) که \lr{epi-graph} این تابع می‌باشد یک مجموعه محدب خواهد بود.
برای اثبات محدب بودن این تابع می‌توان از \lr{hessian} آن استفاده کرده که در تمام نقاط دامنه
یک ماتریس PD می‌باشد.

\section{سوال دوم}
\subsection{الف}
\paragraph{}
یکی از راه‌ها برای بررسی محدب بودن توابع بررسی ماتریس \lr{hessian} آن‌ها است.
اگر این ماتریس یک ماتریس نیمه مثبت معین یا مثبت معین باشد تابع محدب بوده و در غیر این صورت محدب نخواهد بود.

\begin{align}
\begin{split}
	\nabla^2f = \left[\begin{array}{ccc}
		0 & x_3 & x_2 \\
		x_3 & 0 & x_1 \\
		x_1 & x_2 & 0 \\
	\end{array}\right]
\end{split}
\end{align}

\paragraph{}
برای بررسی مثبت معین بودن ماتریس فوق از تعریف استفاده کرده و خواهیم داشت:

\begin{align}
\begin{split}
	\left[\begin{array}{ccc}
		a & b & c \\
	\end{array}\right]
	\left[\begin{array}{ccc}
		0 & x_3 & x_2 \\
		x_3 & 0 & x_1 \\
		x_1 & x_2 & 0 \\
	\end{array}\right]
	\left[\begin{array}{c}
		a \\
		b \\
		c \\
	\end{array}\right]
	\\
	&= 2x_3ab + 2x_2ac + 2x_1bc
\end{split}
\end{align}

\paragraph{}
اگر رابطه‌ی فوق
\(x_1=x_2=x_3=1\)
و
\(a=1, b=1, c=-1\)
حاصل منفی می‌گردد پس ماتریس \lr{hessian}
یک ماتریس مثبت معین نیست و تابع محدب نمی‌باشد.

\subsection{ب}
\paragraph{}
یکی از راه‌ها برای بررسی محدب بودن توابع بررسی ماتریس \lr{hessian} آن‌ها است.
اگر این ماتریس یک ماتریس نیمه مثبت معین یا مثبت معین باشد تابع محدب بوده و در غیر این صورت محدب نخواهد بود.

\begin{align}
\begin{split}
	\nabla^2f = \left[\begin{array}{ccc}
		\frac{2}{x_1^3x_2x_3} & \frac{1}{x_1^2x_2^2x_3} & \frac{1}{x_1^2x_2x_3^2} \\
		\frac{1}{x_1^2x_2^2x_3} & \frac{2}{x_1x_2^3x_3} & \frac{1}{x_1x_2^2x_3^2} \\
		\frac{1}{x_1^2x_2x_3^2} & \frac{1}{x_1x_2^2x_3^2} & \frac{2}{x_1x_2x_3^3} \\
	\end{array}\right]
\end{split}
\end{align}

\paragraph{}
برای بررسی مثبت معین بودن ماتریس فوق از تعریف استفاده کرده و خواهیم داشت:

\begin{align}
\begin{split}
	\left[\begin{array}{ccc}
		a & b & c \\
	\end{array}\right]
	\left[\begin{array}{ccc}
		\frac{2}{x_1^3x_2x_3} & \frac{1}{x_1^2x_2^2x_3} & \frac{1}{x_1^2x_2x_3^2} \\
		\frac{1}{x_1^2x_2^2x_3} & \frac{2}{x_1x_2^3x_3} & \frac{1}{x_1x_2^2x_3^2} \\
		\frac{1}{x_1^2x_2x_3^2} & \frac{1}{x_1x_2^2x_3^2} & \frac{2}{x_1x_2x_3^3} \\
	\end{array}\right]
	\left[\begin{array}{c}
		a \\
		b \\
		c \\
	\end{array}\right]
	\\
	= \frac{1}{x_1x_2x_3}(\frac{a}{x_1} + \frac{c}{x_3})^2
	+  \frac{1}{x_1x_2x_3}(\frac{a}{x_1} + \frac{b}{x_2})^2
	+  \frac{1}{x_1x_2x_3}(\frac{b}{x_2} + \frac{c}{x_3})^2
\end{split}
\end{align}

\paragraph{}
عبارت فوق به صورت مجموع تعدادی مربع کامل با ضرایب مثبت نوشته شده است. بنابراین این عبارت
همواره یک مقدار مثبت خواهد داشت و تابع محدب خواهد بود.

\subsection{ج}
\paragraph{}
این تابع محدب نیست و برای اثبات این موضوع از مثال نقض زیر استفاده می‌کنیم:

\begin{align}
\begin{split}
	x_1 = \left[\begin{array}{c}
		2 \\ 1 \\ 2
	\end{array}\right],
	y_1 = 2^1 + 1^2 = 3
	\\
	x_2 = \left[\begin{array}{c}
		1 \\ 2 \\ 1
	\end{array}\right],
	y_2 = 1^2 + 2^1 = 3
	\\
	\frac{1}{2}x_1 + \frac{1}{2}x_2 =
	\left[\begin{array}{c}
		1.5 \\ 1.5 \\ 1.5
	\end{array}\right],
	f(\frac{1}{2}x_1 + \frac{1}{2}x_2) = 1.5^{1.5} + 1.5^{1.5} = 3.67
	\\
	\frac{1}{2}f(x_1) + \frac{1}{2}f(x_2) = 3
	\\
	f(\frac{1}{2}x_1 + \frac{1}{2}x_2) = 1.5^{1.5} + 1.5^{1.5} = 3.67
	>
	\frac{1}{2}f(x_1) + \frac{1}{2}f(x_2) = 3
\end{split}
\end{align}

\section{سوال سوم}
\subsection{الف}
\paragraph{}
از حل مساله با ابزار \lr{cvx} نتایج زیر حاصل می‌شود:
\begin{align}
	\begin{split}
		x^* = \left[
			\begin{array}{c}
				5 \\ 3
			\end{array}
		\right] \\
		\lambda^* = \left[
			\begin{array}{c}
				0 \\ 2 \\ 0
			\end{array}
		\right]
	\end{split}
\end{align}

\subsection{ب}
\begin{align}
	\begin{split}
		L(x, \lambda)\\
		&= (x_1 - 6)^2 + (x_2 - 4)^2\\
		&+ \lambda_1 (3 - x_1) + \lambda_2 (x_1 + x_2 - 8) + \lambda_3 (x_1 - x_2)\\
		l(\lambda)\\
		&= \underset{x}{\text{min}}L(x, \lambda)\\
		&= 1/4(\lambda_1-\lambda_2+\lambda_3)^2 + 1/4(\lambda_3+\lambda_2)^2 \\
		&+ 1/2(-\lambda_1+\lambda_2-\lambda_3-6)\lambda_1 + 1/2(\lambda_1-2\lambda_2+4)\lambda_2 \\
		&+ 1/2(\lambda_1+2\lambda_3-4)\lambda_3 \\
	\end{split}
\end{align}
\paragraph{}
در نهایت مساله‌ی \lr{dual} به شکل زیر درمیاید.

\begin{equation}
	\begin{aligned}
		& \underset{\lambda}{\text{max}}
		& & l(\lambda) \\
		& & &= 1/4(\lambda_1-\lambda_2+\lambda_3)^2 + 1/4(\lambda_3+\lambda_2)^2 \\
		& & &+ 1/2(-\lambda_1+\lambda_2-\lambda_3-6)\lambda_1 + 1/2(\lambda_1-2\lambda_2+4)\lambda_2 \\
		& & &+ 1/2(\lambda_1+2\lambda_3-4)\lambda_3 \\
		& \text{s.t.} \\
		& & & \lambda_1, \lambda_2, \lambda_3 \ge 0
	\end{aligned}
\end{equation}

\subsection{ج}
\paragraph{}
قضیه \lr{weak duality} بیان می‌کند به ازای هر دو نقطه تابع \lr{dual} مقدار کمتری از تابع \lr{primal} دارد.

\begin{align}
	\begin{split}
		x =
		\left[
		\begin{array}{c}
			3 \\ 5
		\end{array}
		\right]\\
		f(x) = 9 + 1 = 10 \\
		\lambda =
		\left[
		\begin{array}{c}
			0 \\ 0 \\ 0
		\end{array}
		\right]\\
		l(\lambda) = 0 + 0 + 1/2(-6)0 + 1/2(4)0 + 1/2(4)0 = 0 \\
		l(\lambda) < f(x)
	\end{split}
\end{align}

\subsection{د}
مساله‌ی \lr{dual} به فرمت \lr{quadratic} نبوده و به وسیله‌ی \lr{SDPT3}
حل نمی‌شود ولی اگر متغیرهای قسمت اول را در آن جایگذاری کنیم حاصل آن با
حاصل بهینه‌ی مساله‌ی \lr{primal} یکی می‌شود.

\subsection{ذ}
این قضیه برقرار است زیرا متغیرهای \(\lambda_1, \lambda_3\)
متناظر با نامساوی‌هایی که \lr{active} نیستند
برابر با صفر است و مقدار \(\lambda_2\)
در نامساوی دوم غیر صفر می‌باشد که یک نامساری \lr{active} می‌باشد.

\section{سوال چهارم}
\subsection{الف}
\begin{tabular}{l r}
	\(x_i\) & پهنای باند اختصاص‌یافته به تقاضای \(i\)ام\\
	\(f_i(u, v)\) & میزان جریانی که از تقاضای \(i\)ام از لینک \(u, v\) می‌گذرد\\
\end{tabular}
\begin{equation}
	\begin{aligned}
		& \underset{x}{\text{max}}
		& & \sum_{i=1}^{3} \log(x_i) \\
		& \text{s.t.} \\
		& & & \sum_{i=1}^{3} f_i(u, v) \le c(u, v), \forall (u, v) \in {1, 2, 3, 4, 5, 6, 7} \\
		& & & \sum_{v} f_1{u, v} - \sum_{v} f_1{v, u} = \left\{ \begin{array}{c l}
			0 & \forall u \in {2, 3, 4, 6, 7}\\
			x_1 & u = 1\\
			-x_1 & u = 5\\
		\end{array} \right.\\
		& & & \sum_{v} f_2{u, v} - \sum_{v} f_2{v, u} = \left\{ \begin{array}{c l}
			0 & \forall u \in {1, 3, 4, 5, 7}\\
			x_2 & u = 2\\
			-x_2 & u = 6\\
		\end{array} \right.\\
		& & & \sum_{v} f_3{u, v} - \sum_{v} f_3{v, u} = \left\{ \begin{array}{c l}
			0 & \forall u \in {1, 2, 3, 5, 7}\\
			x_3 & u = 4\\
			-x_3 & u = 6\\
		\end{array} \right.\\
		& & & x_1, x_2, x_3 \ge 0 \\
		& & & f_i(u, v) \ge 0, \forall (u, v) \in {1, 2, 3, 4, 5, 6, 7}, \forall i \in {1, 2, 3}
	\end{aligned}
\end{equation}

\subsection{ب}

\end{document}